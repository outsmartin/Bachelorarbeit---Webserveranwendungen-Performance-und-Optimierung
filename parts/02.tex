\part{Theoretische Grundlagen}
\label{sec:theory}

\subsection{Usability}
Web Performance hat einen großen Einfluss auf die Bedienbarkeit von Webseiten. Ohne die starke Verbesserung ihrer Performance hätte Google mit GMail und ihren anderen Web Applikationen keinen Erfolg erzielen können. Nur wenn eine Anwendung auch in einer akzeptablen Geschwindigkeit auf Benutzerinteraktionen reagieren kann, hat sie eine Chance auf dem Markt zu bestehen. Es gibt einige Studien, die sich mit Web Usability auseinandersetzen. Zu den Ergebnissen gehörte unter anderem, dass langsamere Webseiten zu Vertrauensverlust sowie Nutzerfrustration führen. Als weitere Konsequenz ist der empfundene Qualitätsverlust zu nennen. Die genannten Punkte führen letztendlich zu niedrigeren Konversionsraten, das heißt, dass ...+ höheren Bailoutraten (Absprungrate)

%TODO niedrigere Konversionsraten und Bailoutraten erklären und Satz beenden

%TODO quelle?
\subsection{Google Ranking}
Für (Internet-)Firmen ist es von immenser Bedeutung gefunden zu werden. Ein Großteil der Internetnutzer sucht Angebote über die Google-Suche. Google hat mit ihrer Initiative ``Let's make the web faster'' für viel Entwicklung und Aktivität im Bereich Web Performance gesorgt und arbeitet zielgerichtet weiter in diese Richtung. Dazu gehört seit  2008 Google AdWords und seit 2010 die Google Suche selbst. Dies stellt viele Unternehmen vor die Aufgabe, ihre Webseite zu optimieren und zu beschleunigen, um im Google Ranking konkurrenzfähig zu bleiben. Wie genau Google die Webseiten testet, ist nur zum Teil bekannt, da diese Informationen zu Googles Geschäftsgeheimnissen gehören. Es finden sich aber einige Fakten, die bei der Optimierung für die Google Suche, im Bereich Web Performance, helfen können.
Bekannt ist, dass der Google Suchmaschinen Bot nichts mit der Geschwindigkeitsmessung zu tun hat und Google nur Daten von echten Nutzern, die die Google Toolbar in ihrem Browser installiert haben, nutzt. Leider sind Daten nur für Internet Explorer ab Version 6 und Firefox ab Version 2 verfügbar. Als Kriterium für die Bewertung wird die Onload Zeit gemessen. Dabei führt das verzögerte Nachladen von Inhalten zu einem besseren Ergebnis. Wenn eine Webseite für Google optimiert werden soll, muss also auch die Performance mit berücksichtigt werden. Performance ist aber nur ein Teil der Google Bewertung. Für Webseitenbetreiber gilt daher, dass aktuelle und gut strukturierte Inhalte nicht hinter die Performance gestellt werden dürfen.

%TODO bild?
%%http://www.webperformancetoday.com/2011/08/05/faqs-google-seo-search-ranking-website-speed/

\subsection{Serverlast}
Eine umfassende Verbesserung der Auslieferungszeiten von Webseiten hat direkten Einfluss auf die Serverperformance. Dies ist leicht in einem Experiment feststellbar, wie folgende Überlegung zeigt: Wenn eine Webseite nur noch die Hälfte der Zeit benötigt, um ausgeliefert zu werden, hat man doppelt soviel Auslieferungskapazität zur Verfügung. Das spart Kosten für Server und Traffic. Außerdem ermöglichen Caching und Optimierungen einen Großteil an Prozessorlast einzusparen, der dann für andere Aufgaben zur Verfügung steht. Besonders wichtig ist die Performance in Momenten hoher Zugriffszahlen, beispielsweise wenn eine Webseite in einem prominenten Newsportal erwähnt wird. Oftmals bricht dann die Webseite zusammen, weil die Administratoren nicht mit solch einem Ansturm gerechnet haben. Erwähnenswert ist das Newsportal www.heise.de auf dem regelmäßig davon gesprochen wird, es wurde eine Webseite ``geheist''. Noch problematischer wird es, wenn ein Unternehmen Ziel krimineller Aktivitäten wird. Aktuell zu beobachten ist dieses Phänomen bei den Angriffen des Miner-Botnetzes auf mehrere deutsche Webseiten. Um sich vor solchen Distributed Denial of Service Angriffen schützen zu können, ist eine Performante Webseite sehr wichtig, um genug Ressourcen für Firewalls und andere Gegenmaßnahmen zur Verfügung zu haben.

%TODO ``geheist'' erklären

%\section{Performance Bottlenecks}
\section{Einordnung von Performanzproblemen}
Performance Bottlenecks oder auch Performance Flaschenhälse genannt, beschreiben Schwachpunkte in einem Datenverarbeitungssystem. Auf Web Performance angewendet, kann man das in clientseitige Probleme sowie serverseitige Probleme unterteilen.
%TODO Begriffe ueberdenken

\subsection{Clientseitige Bottlenecks}
\subsubsection{Netzwerkverbindung}
Die Netzwerkverbindung kann auf zwei Ebenen für Probleme sorgen. Zum einen kann eine hohe Latenz für langsame auch auf Serverseite auftreten, aber da kann mit mehr Servern und besseren Anbindungen Abhilfe geschafft werden. Viel problematischer wird das auf der Clientseite, da selten bekannt ist, mit welcher Anbindung Nutzer eine Webseite besuchen werden.%TODO bild netzwerk
% Ausdruck! TODO

Da in Deutschland der Breitbandausbau noch lange kein Stadium erreicht hat, der es Webseitenbetreibern erlauben würde auf die Größe der Webseite, die sie an ihre Betrachter ausliefern keinen Wert zu legen. Der Breitbandausbau peilt für 2011 eine 100 prozentige Versorgung mit 1 Mbit/s an und so sind noch viele Nutzer sind mit langsamen Internetzugängen im Internet unterwegs. %TODO Fussnote mit Link

\subsubsection{Browser}
Als Schnittstelle zwischen Mensch und Webseite ist der Browser ein besonders kritischer Punkt und muss bei Web Performanceanalysen besonders beachtet werden. Zum einen müssen die verschiedenen Browserfamilien, mit ihren Eigenheiten und Problemen und zum anderen die Tatsache das ein Browser nur so schnell arbeiten kann, wie der Computer auf dem er ausgeführt wird zulässt, beachtet werden. 
%TODO 2 Saetze draus machen
Problemzonen sind:
\begin{itemize}
  \item Anzahl von zu berechnenden DOM-Elementen %TODO was ist DOM
  \item Javascript Ausführungszeiten
  \item Anzahl benötigter HTTP Zugriffe %z.B. Stylesheets Bilder  Warum ist das ein Problem
\end{itemize}

\subsubsection{DOM Elemente}
\subsection{Serverseitig}
% moderne dynamische Webanwendungen sind komplexe Applikationen


\subsubsection{Datenbankabfragen}
So gut wie jede Website nutzt mittlerweile Datenbanken zur Verwaltung ihrer Datenbeständen.

Oft sind diese Datenbanken ein Schwachpunkt für die Web-Performance, da meistens Daten von der Festplatte gelesen werden müssen und komplexe Abfragen viel Zeit in Anspruch nehmen können.

% (Tablelocking Nutzer gleichzeitig)
\subsubsection{Skriptausführung}
\subsubsection{Skriptkompilierung}

\section{Technologiestack und Optimierungsmöglichkeiten}
Im Rahmen dieser Bachelorarbeit werden verschiedenste Technologien verwendet und auf ihnen aufbauende Werkzeuge zu Hilfe genommen. Um zu verstehen wo Probleme lokalisiert sind und wie solche Schwachstellen zu finden sind muss man sich mit dem vorhandenem Technologiestack auseinandersetzen und ihn analysieren.
\begin{description}
  \item[Server] Ein Server ist ein Computer, der Informationen und Dienste für andere Computer zur Verfügung stellt
  \item[Betriebssystem] Die Software die auf dem Server läuft. In der pludoni GmbH wird das Linuxderivat Debian eingesetzt.
  \item[Datenbank] Als Datenbank wird eine strukturierte Sammlung von Daten bezeichnet. Einer der häufigsten Vertreter, gerade im Zusammenhang mit PHP-Webanwendungen, ist MySQL
  \item[Web Server] Der Web Server ist für die zuverlässige Auslieferung von Webseiten zuständig. Er beantwortet die Anfragen die die Nutzer mit dem Browser stellen. Apache2 wird dafür in der pludoni GmbH eingesetzt.
  \item[PHP] ist eine dynamische typisierte Programmiersprache, zur Erstellung von Webanwendungen. %TODO Features von PHP, 
  \item[Drupal] Drupal ist ein CMS und Framework, welches in PHP geschrieben ist und viele direkt nutzbare Funktionen mitbringt. Dazu gehören unter anderem Administrationsoberflächen, Newsaggregatoren, Veröffentlichungsabläufe für Artikel und Blogbeiträge sowie Suchmaschinenoptimierungen. Außerdem ermöglicht Drupal die Installation weiterer, durch Nutzer entwickelte, Module. Dadurch wird eine fast unbegrenzte Funktionsvielfalt ermöglicht. Im vorliegenden Fall wird Drupal in der Version 5 eingesetzt.
  \item[Client] Im Browser der Nutzer kommt am Ende HTML an, dies wird durch Drupal generiert und vom Webserver ausgeliefert. Dabei werden Bilder verarbeitet Javascript Programme ausgeführt und andere Elemente wie Flashapplikationen und Videos berechnet. 
\end{description}

%TODO Bildchen vom Ablauf!

\subsection{Server}
\subsection{PHP}
\subsection{Drupal (5)}
Drupal ist ein Content-Management-System, dass heißt es ermöglicht dem Nutzer eine dynamische Webseite zu erstellen ohne spezielle Programmierkenntnisse zu besitzen. 

% Ist sowohl CMS als auch Framework
\subsection{Benchmark - Werkzeuge}
\subsection{Debugger / Profiler}
\section{Methoden der Web-Performance Optimierung}
Die Methoden zur Leistungssteigerung können wie im Technologiestack untergliedert werden und jedes Element muss für sich betrachtet werden.

\begin{description}
  \item[Serverhardware] Der einzige Parameter der am Server verbessert werden kann ist seine Leistung, dass heißt Datendurchsatz und Rechengeschwindigkeit. Dies geschieht durch CPU Upgrades oder Arbeitsspeichererweiterung. Weitergehend kann man die Festplattenzugriffsgeschwindigkeiten durch RAID Verbünde und neue Technologien wie SSD Speicher verbessern. 
  % Replikation also mehrere Server Load Balancer
  \item[Betriebssystem] Ansatzpunkte für Verbesserungen auf Betriebssystemebene sind Memory-Mapping, dass heißt Speicherbereiche die normalerweise auf der Festplatte liegen werden in den Arbeitsspeicher umgelagert um die Latenzen zu verringern. Dies wird genutzt um Caches zu beschleunigen, die normalerweise von der Festplatte lesen.
  \item[Datenbank] Auf Datenbankebene kann im Fall von MySQL nur beschränkt optimiert werden. Zum einen sind Indizes anzulegen bei häufig genutzten Tabellen und zum anderen kann man MyISAM statt InnoDB nutzen um performanter zu sein.
  % Replikation also mehrere DB Server
  \item[Web Server] Optimierungen am Webserver sind sehr schwierig da die Webserver Performance hauptsächlich von der Serverleistung abhängt. Die Anzahl der gleichzeitigen Zugriffe wird maßgeblich durch den verfügbaren Arbeitsspeicher und die verfügbare Bandbreite eingeschränkt.
  \item[PHP] 
    % Neukomilierung verhindern
    % Clevere Programmierung, Wahl der Frameworks
  \item[Drupal] 
    % Caching 
    % Anzahl der Module, 
    % CSS Aggregation
  \item[HTML] 
    % Caching 
\end{description}


%
\part{Praxisteil}

Versuchsaufbau:

Testplattform: pludoni Server eq4
Prozessor: Intel® Core™ i7-920
Arbeitsspeicher: 8 GB DDR3 RAM
Festplatten: 2 x 750 GB SATA-II HDD (Software-RAID 1)
Netzwerkverbindung: 100Mbit

Livesystem: pludoni Server 
Prozessor: Intel® Core™2 Quad CPU Q6600 @ 2.4 Ghz
Arbeitsspeicher: 4 GB DDR3 RAM
Festplatten: 2 x 750 GB SATA-II HDD (Software-RAID 1) %TODO aktualisieren
Netzwerkverbindung: 100Mbit

Testclient: Virtualbox
Firefox 5

Testobjekt:
Itsax Startseite


Ausgangszustand:
Sreeram Ramachandran ein Software Engineer bei der Firma Google hat eine Analyse über 4.2 Milliarden Seiten veröffentlicht. Diese ist im Rahmen der Initiative ``Let's make the web faster'' entstanden und zeigt häufige Fehlerquellen und ungenutztes Potential auf. Die durchschnittliche Webseite hat laut Ramachandran 320 KB Größe, 44 verschiedene Ressourcen und es werden nur 66\% der komprimierbaren Inhalte tatsächlich komprimiert.
Itsax.de hat 106 Ressourcen und 444 KB an Daten. Schon anhand dieser zwei Zahlen lässt sich eine vergleichsweise schwache Leistung vorhersehen. Besonders die Anzahl an verschiedenen Ressourcen deutet auf Missstände hin, da Parallelisierung von Zugriffen nur bis zu einem bestimmten Grad möglich ist. Die Time to First Byte(TtFB) von 674ms bezeichnet die Zeit die vergeht bis der Webbrowser die ersten Daten empfängt, dass bedeutet aber noch nicht das der Nutzer schon Inhalte präsentiert bekommt. Die Inhalte werden erst angezeigt nachdem die Time to Start Render(TtSR) vergangen ist, der Nutzer muss demnach ungefähr zwei Sekunden warten bis die Webseite im Browser anfängt sich aufzubauen. Die Load Time(LT) bezeichnet dann die Zeit die vergeht bis die Seite komplett dargestellt wird und der Benutzer sie ohne Einschränkungen bedienen kann. Es können  aber auch nach der LT noch Inhalte nachgeladen werden, wie zum Beispiel gestreamte Videos oder andere asynchrone Inhalte. Diese nachgeladenen Inhalte wirken sich aber nicht mehr Negativ auf die User Experience aus, solange sie im Rahmen bleiben und nicht wichtige Teile der Webseite wie zum Beispiel das Hauptmenü noch per Flash geladen werden müssen. Die Anzahl der DOM Elemente bezeichnet alle vom Browser zu verarbeitenden Objekte und ist ein Indikator für die Komplexität der Webseite, je mehr Elemente also vorhanden sind desto länger muss der Browser die Positionierung und Darstellung berechnen. Die Inhalte auf der Seite Itsax.de sind in Abbildung ? dargestellt, einmal im Bezug auf Größe und einmal aufgeschlüsselt nach der Anzahl der benötigten Requests um die Inhalte vom Server anzufordern. 

\includegraphics[scale=0.5]{material/start_request_pie.png}
\begin{tabbing}
Request \quad\= blablabla \quad\= \kill
\textbf{Typ} 	 \> \textbf{Anzahl} \\
text/css	 \>	24 	\\
image/gif	 \>	23 	\\
image/jpeg	 \>	22 	\\
javascript	 \>	20 	\\ 
image/png	 \>	15 	\\
text/html	 \>	2 	\\

\end{tabbing}

\includegraphics[scale=0.5]{material/start_byte_pie.png}

%http://code.google.com/intl/de/speed/articles/web-metrics.html

%TODO Bilder einfügen aus materials

text/css24
image/gif	22
image/jpeg	22
javascript	20
image/png	15
text/html	3

image/jpeg	142251
image/png	101428
javascript	84758
image/gif	73378
text/css	30222
text/html	29917
Ausgangssituation:
Server Software:        Apache/2.2.16
Server Hostname:        itsax.it-jobs-und-stellen.de
Server Port:            80

Document Path:          /
Document Length:        65218 bytes

Concurrency Level:      1
Time taken for tests:   18.182 seconds
Complete requests:      50

Write errors:           0
Total transferred:      3266726 bytes
HTML transferred:       3239226 bytes
Requests per second:    2.75 [\#/sec] (mean)
Time per request:       363.640 [ms] (mean)
Time per request:       363.640 [ms] (mean, across all concurrent requests)
Transfer rate:          175.46 [Kbytes/sec] received




Mit APC aktiviert:
Server Software:        Apache/2.2.16
Server Hostname:        itsax.it-jobs-und-stellen.de
Server Port:            80

Document Path:          /
Document Length:        64842 bytes

Concurrency Level:      1
Time taken for tests:   11.681 seconds
Complete requests:      50

Write errors:           0
Total transferred:      3261730 bytes
HTML transferred:       3234230 bytes
Requests per second:    4.28 [\#/sec] (mean)
Time per request:       233.620 [ms] (mean)
Time per request:       233.620 [ms] (mean, across all concurrent requests)
Transfer rate:          272.69 [Kbytes/sec] received

Mit Drupal Cache in normal Einstellung:
Server Software:        Apache/2.2.16
Server Hostname:        itsax.it-jobs-und-stellen.de
Server Port:            80

Document Path:          /
Document Length:        65005 bytes

Concurrency Level:      1
Time taken for tests:   2.082 seconds
Complete requests:      50
Failed requests:        0
Write errors:           0
Total transferred:      3276900 bytes
HTML transferred:       3250250 bytes
Requests per second:    24.02 [\#/sec] (mean)
Time per request:       41.638 [ms] (mean)
Time per request:       41.638 [ms] (mean, across all concurrent requests)
Transfer rate:          1537.09 [Kbytes/sec] received

Mit Drupal Cache in aggressiv Einstellung:



Test von Serverunabhängigen verbesserungen:
über das Analysetool von webpagetest.org
Testverfahren: 10 konsekutive Tests (maximum) der Durschnittlichste Testlauf wird betrachtet
Teststandort Frankfurt a. M.
Browser: IE9
Connection: DSL 1,5 Mbps / 50ms RTT
Only First View




Load Time: 3.728
Time to First Byte: 0.674s 	
Time to Start Render: 2.002s
\#DOM Elements: 855 	
\#Requests: 106
Bytes In: 444 KB
%http://www.webpagetest.org/result/110809_BH_198S4/


Im folgenden Abschnitt werden die eingesetzten Methoden dargestellt und ihre Auswirkungen auf die Webperformance der Seite ITsax.de dargestellt. 
JS Aggregation und Minifizierung mit dem Javascript Aggregation Modul:
Aggregation und Minifizierung sind Verfahren, die in der Webperformance Optimierung häufig eingesetzt werden. Für das Drupal 5 System gibt es fertige Module die diese Aufgabe übernehmen. Das Modul interveniert innerhalb des Drupal Kerns und ersetzt die Javascripts, die vorher direkt so ausgegeben wurden wie sie die Module lieferten, durch eine einzelne, das heisst aggregierte Version, die optional minifiziert werden kann. Diese Minifizierung wird natürlich genutzt und spart einige KB. Minifizierung wird ermöglicht durch die Nutzung von JSmin https://github.com/rgrove/jsmin-php/. JSmin ist ein Filter der unter anderem Kommentare entfernt und mehrere Leerzeichen zu einem zusammenfasst. Durch die Nutzung dieser Methode konnten 11 HTTP Abfragen und 11 KB an Datenvolumen eingespart werden.
Load Time: 3.658s
Time to First Byte: 0.595s
Time to Start Render: 1.915s
\#DOM Elements: 844 	
\#Requests: 95 %!
Bytes In: 432 KB
%http://www.webpagetest.org/result/110809_Z1_198TK/

CSS Aggregation und Komprimierung:
Analog zur Javascript Optimierung kann man auch die CSS Aggregation betrachten. Es werden durch Zusammenfassung der einzelnen CSS Dateien HTTP Abfragen eingespart. Wie man an der gesunkenen Anzahl der Requests sehen kann, wurden 21 Abfragen nur durch Zusammenfügen der einzelnen CSS Dateien zu einer einzigen eingespart. Durch die Bereinigung beziehungsweise Komprimierung werden dabei außerdem 17 Kb an überflüssigen Zeichen entfernt. 
Load Time: 3.577s
Time to First Byte: 0.649s
Time to Start Render: 1.577s
\#DOM Elements: 834 	
\#Requests: 85 %!
Bytes In: 427 KB
%http://www.webpagetest.org/result/110809_XB_198VE/

Drupal Boost Module:
Load Time: 3.233s
Time to First Byte: 0.172s %!
Time to Start Render: 1.473s
\#DOM Elements: 856 	
\#Requests: 106 %!
Bytes In: 444 KB
%http://www.webpagetest.org/result/110809_AP_198XJ/

Bildoptimierungen mit jpegoptim und OptiPNG verlustfrei!: 
Bilder und Grafiken bieten oft großen Optimierungsspielraum, zum einen durch die richtige Auswahl der Dateiformate und zum anderen durch Komprimierung der Bilder. Da es auf www.ITsax.de nicht nur statische Inhalte gibt, sondern auch durch Communitymitglieder und Communitymanager eingestellte Inhalte verwaltet werden müssen, sollte eine nachträgliche Qualitätsoptimierung der hochgeladenen Bilder durchgeführt werden. Um dies umzusetzen sind die Programme OptiPNG und jpegOptim zu empfehlen. Auf jeden Fall sollte eine verlustfreie Komprimierung durchgeführt werden, da die Bilder in diesem Fall nur an Dateigröße verlieren und die Bildqualität unberührt bleibt. Da es sich bei beiden Programmen um Kommandozeilenprogramme handelt kann man ihre Anwendung leicht automatisieren. Mit dem Linuxbefehl find, der praktischerweise eine Möglichkeit Befehle auszuführen besitzt, kann man direkt die entsprechenden Dateien an die Optimierer übergeben. Diese Aktionen können dann über einen Cronjob periodisch jede Nacht ausgeführt werden. Die Befehle sehen dann wie folgt aus: 
find . -name "*.png" -exec optipng -o7 {} \;
find . -name "*.jpg" -exec jpegoptim {} \;

%%http://optipng.sourceforge.net/
%%http://www.kokkonen.net/tjko/projects.html

Load Time: 3.640
Time to First Byte: 0.636s %!
Time to Start Render: 1.894s
\#DOM Elements: 855 	
\#Requests: 106 %!
Bytes In: 429 KB % 15kb gespart
%http://www.webpagetest.org/result/110809_C4_1999C/

Bildoptimierungen mit OptiPNG verlustfrei und mit jpegoptim verlustbehaftet! 50\% kompression:
find . -name "*.png" -exec optipng -o7 {} \;
find . -name "*.jpg" -exec jpegoptim {} \;

Load Time: 3.255s
Time to First Byte: 0.669s %!
Time to Start Render: 1.908s
\#DOM Elements: 855 	
\#Requests: 106 %!
Bytes In: 371 KB % 73kb gespart
%http://www.webpagetest.org/result/110809_15_199GY/

Drupal 5 Fehler bei umgefärbten Themes:
Das Framework Drupal 5 benutzt Themes zu Gestaltung der Oberfläche, um diese farblich anpassen zu können wurde das Color Modul installiert, welches Themeveränderungen ermöglicht. Dadurch das das Theme nur kopiert und die Farben geändert werden entstehen bei diesem Vorgang unnötige Duplikate die beim Laden der Seite mitgeschleppt werden. Um diese zu Entfernen wird einfach das Standardtheme durch das Modifizierte ersetzt. Dafür müssen  nur noch einige Pfade in der style.css angepasst werden und man spart in dem fall von ITsax.de 4 kb, was immerhin ca 1\% der übertragenen Datenmenge ausmacht.
händisch gemergte styles
style auf standard setzen
vorher die images und das geänderte stylesheet kopieren
Load Time: 3.626s
Time to First Byte: 0.629s %!
Time to Start Render: 1.890s
\#DOM Elements: 854 	
\#Requests: 104 %!
Bytes In: 440 KB % 4kb gespart
%http://www.webpagetest.org/result/110809_SZ_19APH/

Theme Bilder Spriten:
Spriting wurde ursprünglich in der Videospielentwicklung verwendet um Bilder in der Grafikspeicher zu laden. In der Webentwicklung ist es eine effektive Technik um Bilder ohne mehrfachen Overhead zu laden. Beim Spriting wird aus vielen einzelnen Bildern ein einziges Bild erstellt, dass anstelle der vielen Bilder geladen wird. Um die Bilder dann noch einzeln Anzeigen zu können werden CSS Befehle genutzt, die es ermöglichen die Größe und die Position eines Bildausschnittes anzuzeigen. 
Load Time: 3.707
Time to First Byte: 0.669s %!
Time to Start Render: 1.968s
\#DOM Elements: 855 	
\#Requests: 103 
Bytes In: 443 KB 
%http://www.webpagetest.org/result/110810_FZ_19GBA/

Verschiedene Module von Startseite entfernen:
Um zu überprüfen welchen Einfluss verschiedene, im Netzwerkgraphen auffällige Module auf die Gesamtperformance haben werden sie Testweise komplett deaktiviert. So kann man entscheiden bei welchen Module zusätzlicher Programmieraufwand lohnenswert ist.
Partnerslideshow:
Load Time: 2.695s
Time to First Byte: 0.636s %!
Time to Start Render: 1.838s
\#DOM Elements: 790 	
\#Requests: 80 
Bytes In: 276 KB
%http://www.webpagetest.org/result/110810_JT_19HS5/

facebook fenster:
Load Time: 3.478s
Time to First Byte: 0.668s %!
Time to Start Render: 1.966s
\#DOM Elements: 779 	
\#Requests: 95 
Bytes In: 406 KB 
%http://www.webpagetest.org/result/110810_63_19HZT/

Jobleiste deaktiviert:
Load Time: 3.367s
Time to First Byte: 0.617s %!
Time to Start Render: 1.709s
\#DOM Elements: 822 	
\#Requests: 94 
Bytes In: 411 KB
%http://www.webpagetest.org/result/110810_4F_19J8H/

Umprogrammierung verschiedener Module: Um die langsamen Module weiterhin nutzen zu können muss eine Lösung gefunden werden, die es ermöglicht Inhalte nachzuladen nachdem die Seite komplett geladen wurde. Um das zu erreichen kann man mit der Hilfe von Javascript bestimmte DOM Elemente über einen Timeoutbefehl erst nachdem der Browser gemeldet hat er hat die Seite fertig geladen, die Inhalte nachladen.
Partnerslideshow:
Load Time: 2.749s (3.794)
Time to First Byte: 0.626s %!
Time to Start Render: 1.842s
\#DOM Elements: 855 	
\#Requests: 82 (107)
Bytes In: 284 KB (395)
%http://www.webpagetest.org/result/110810_3Z_19MRE/

facebook fenster:
Load Time: 3.411s (4.258)
Time to First Byte: 0.576s %!
Time to Start Render: 1.848s
\#DOM Elements: 857
\#Requests: 95 (106)
Bytes In: 406 KB (444)
%http://www.webpagetest.org/result/110810_AJ_19N72/

Partneranzeigen:
Load Time: 3.871s (5.759s)
Time to First Byte: 0.669s %!
Time to Start Render: 2.255s
\#DOM Elements: 857 	
\#Requests: 105 (108)
Bytes In: 417 KB (448)
%http://www.webpagetest.org/result/110815_N6_1AWKT/1/details/


Komplett alle Maßnahmen:

%\input{parts/02-2}
