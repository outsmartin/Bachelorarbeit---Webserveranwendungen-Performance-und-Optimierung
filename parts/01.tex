\part{Einleitung}

\label{sec:intro}
\section{Motivation}

Der Autor dieser Arbeit absolvierte sein Pflichtpraktikum in der pludoni GmbH und war bis zum heutigen Tag als Werksstudent tätig.
\subsection{pludoni GmbH}
“pludoni kommt aus dem Esperanto und bedeutet weitergeben.” Das Unternehmen wurde 2008 von Dr. Jörg Klukas gegründet und hat als Ziel den regionalen Mittelstand bei der Kommunikation und Bewerbersuche durch Communities zu unterstützen. Die von pludoni gemanagten Communities heißen ITsax.de, ITmitte.de und MINTsax.de. Sie richten sich an Technologie-Firmen aus Sachsen beziehungsweise Mitteldeutschland und versuchen über zahlreiche Dienstleistungen einen Mehrwert für die beteiligten Unternehmen zu schaffen, im Allgemeinen gehören dazu:

\begin{itemize}
 \item Aggregation, Veröffentlichung und Weitergabe von Stellenanzeigen
 \item Organisation von regelmäßigen Community-Treffen zum Austausch der Unternehmen,
 \item Unterstützung und Beratung bei der Gestaltung Suchmaschinen-optimierter Inhalte und
 \item Infrastruktur zur gegenseitigen Empfehlung von Fachkräften und Lernenden.
\end{itemize}

\subsection{Ziel}
Alle Communities sind zum heutigen Zeitpunkt durch das Content Management System Drupal 5 realisiert. Drupal 5 ist mittlerweile schon vier Jahre alt und nicht mehr auf dem neuesten Entwicklungsstand der Web Performance. Es gibt Drupal mittlerweile schon in der Version 7.7 und es sind viele Verbesserungen in Sachen Geschwindigkeit eingeflossen. Leider ist es nicht trivial ein Upgrade durchzuführen, da viele Strukturen sich stark verändert haben. Aus diesem Grund wurde der Autor damit beauftragt, die Möglichkeiten einer Drupal 5 Optimierung am Beispiel der ältesten Community ITsax.de zu ergründen und, wenn vorhanden, durchzuführen. Dabei soll ein Leitfaden entwickelt werden, anhand dessen Drupal 5 und allgemein Webseiten effizient optimiert werden können.

\section{Aufbau}
Die Arbeit ist in zwei Teile aufgeteilt. Der Theorieteil beschäftigt sich mit den Grundlagen der Web Performance beziehungsweise der Web Performance Optimierung. Im Anschluss daran werden im Praxisteil Performancemängel der Webseite itsax.de untersucht und, sofern vorhanden, behoben. 
